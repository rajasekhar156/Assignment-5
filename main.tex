\documentclass[journal,12pt,twocolumn]{IEEEtran}

\usepackage{setspace}
\usepackage{gensymb}
\singlespacing
\usepackage[cmex10]{amsmath}

\usepackage{amsthm}

\usepackage{mathrsfs}
\usepackage{txfonts}
\usepackage{stfloats}
\usepackage{bm}
\usepackage{cite}
\usepackage{cases}
\usepackage{subfig}

\usepackage{longtable}
\usepackage{multirow}

\usepackage{enumitem}
\usepackage{mathtools}
\usepackage{steinmetz}
\usepackage{tikz}
\usepackage{circuitikz}
\usepackage{verbatim}
\usepackage{tfrupee}
\usepackage[breaklinks=true]{hyperref}
\usepackage{graphicx}
\usepackage{tkz-euclide}

\usetikzlibrary{calc,math}
\usepackage{listings}
    \usepackage{color}                                            %%
    \usepackage{array}                                            %%
    \usepackage{longtable}                                        %%
    \usepackage{calc}                                             %%
    \usepackage{multirow}                                         %%
    \usepackage{hhline}                                           %%
    \usepackage{ifthen}                                           %%
    \usepackage{lscape}     
\usepackage{multicol}
\usepackage{chngcntr}

\DeclareMathOperator*{\Res}{Res}

\renewcommand\thesection{\arabic{section}}
\renewcommand\thesubsection{\thesection.\arabic{subsection}}
\renewcommand\thesubsubsection{\thesubsection.\arabic{subsubsection}}

\renewcommand\thesectiondis{\arabic{section}}
\renewcommand\thesubsectiondis{\thesectiondis.\arabic{subsection}}
\renewcommand\thesubsubsectiondis{\thesubsectiondis.\arabic{subsubsection}}


\hyphenation{op-tical net-works semi-conduc-tor}
\def\inputGnumericTable{}                                 %%

\lstset{
%language=C,
frame=single, 
breaklines=true,
columns=fullflexible
}
\begin{document}


\newtheorem{theorem}{Theorem}[section]
\newtheorem{problem}{Problem}
\newtheorem{proposition}{Proposition}[section]
\newtheorem{lemma}{Lemma}[section]
\newtheorem{corollary}[theorem]{Corollary}
\newtheorem{example}{Example}[section]
\newtheorem{definition}[problem]{Definition}

\newcommand{\BEQA}{\begin{eqnarray}}
\newcommand{\EEQA}{\end{eqnarray}}
\newcommand{\define}{\stackrel{\triangle}{=}}
\bibliographystyle{IEEEtran}
\raggedbottom
\setlength{\parindent}{0pt}
\providecommand{\mbf}{\mathbf}
\providecommand{\pr}[1]{\ensuremath{\Pr\left(#1\right)}}
\providecommand{\qfunc}[1]{\ensuremath{Q\left(#1\right)}}
\providecommand{\sbrak}[1]{\ensuremath{{}\left[#1\right]}}
\providecommand{\lsbrak}[1]{\ensuremath{{}\left[#1\right.}}
\providecommand{\rsbrak}[1]{\ensuremath{{}\left.#1\right]}}
\providecommand{\brak}[1]{\ensuremath{\left(#1\right)}}
\providecommand{\lbrak}[1]{\ensuremath{\left(#1\right.}}
\providecommand{\rbrak}[1]{\ensuremath{\left.#1\right)}}
\providecommand{\cbrak}[1]{\ensuremath{\left\{#1\right\}}}
\providecommand{\lcbrak}[1]{\ensuremath{\left\{#1\right.}}
\providecommand{\rcbrak}[1]{\ensuremath{\left.#1\right\}}}
\theoremstyle{remark}
\newtheorem{rem}{Remark}
\newcommand{\sgn}{\mathop{\mathrm{sgn}}}
\providecommand{\abs}[1]{\(\left\vert#1\right\vert\)}
\providecommand{\res}[1]{\Res\displaylimits_{#1}} 
\providecommand{\norm}[1]{\(\left\lVert#1\right\rVert\)}
%\providecommand{\norm}[1]{\lVert#1\rVert}
\providecommand{\mtx}[1]{\mathbf{#1}}
\providecommand{\mean}[1]{E\(\left[ #1 \right]\)}
\providecommand{\fourier}{\overset{\mathcal{F}}{ \rightleftharpoons}}
%\providecommand{\hilbert}{\overset{\mathcal{H}}{ \rightleftharpoons}}
\providecommand{\system}{\overset{\mathcal{H}}{ \longleftrightarrow}}
	%\newcommand{\solution}[2]{\textbf{Solution:}{#1}}
\newcommand{\solution}{\noindent \textbf{Solution: }}
\newcommand{\cosec}{\,\text{cosec}\,}
\providecommand{\dec}[2]{\ensuremath{\overset{#1}{\underset{#2}{\gtrless}}}}
\newcommand{\myvec}[1]{\ensuremath{\begin{pmatrix}#1\end{pmatrix}}}
\newcommand{\mydet}[1]{\ensuremath{}}
\numberwithin{equation}{subsection}
\makeatletter
\@addtoreset{figure}{problem}
\makeatother
\let\StandardTheFigure\thefigure
\let\vec\mathbf
\renewcommand{\thefigure}{\theproblem}
\def\putbox#1#2#3{\makebox[0in][l]{\makebox[#1][l]{}\raisebox{\baselineskip}[0in][0in]{\raisebox{#2}[0in][0in]{#3}}}}
     \def\rightbox#1{\makebox[0in][r]{#1}}
     \def\centbox#1{\makebox[0in]{#1}}
     \def\topbox#1{\raisebox{-\baselineskip}[0in][0in]{#1}}
     \def\midbox#1{\raisebox{-0.5\baselineskip}[0in][0in]{#1}}
\vspace{3cm}
\title{AI1103 - Assignment 5}
\author{I.Rajasekhar Reddy -- CS20BTECH11020}
\maketitle
\newpage
\bigskip
\renewcommand{\thefigure}{\theenumi}
\renewcommand{\thetable}{\theenumi}
Download all latex codes from
%
\begin{lstlisting}
https://github.com/rajasekhar156/Assignment-5/blob/main/main.tex
\end{lstlisting}
\section*{QUESTION}
Let X$_{1}$ and X$_{2}$ be a random sample of size two
from a distribution with probability density
function
\begin{align}
    f_{\theta}(x) &= \theta \brak{\dfrac{1}{\sqrt{2\pi}}}e^{-\dfrac{1}{2} x^{2}} + \brak{1-\theta}\brak{\dfrac{1}{2}}e^{-\mid x\mid} \nonumber,
\end{align}
$-\infty<x<\infty$,\\
where  $\theta \in \cbrak{ 0,\dfrac{1}{2}, 1 }$. If the observed values
of X$_{1}$ and X$_{2}$ are 0 and 2, respectively, then
the maximum likelihood estimate of $\theta$ is
\begin{enumerate}
    \item 0 
    \item $\frac{1}{2}$
    \item 1
    \item not unique
\end{enumerate}
\section*{ANSWER}
Given X$_{1}=$0, X$_{2}=$2, n=2 and
\begin{align}
    f_{\theta}(x) =\theta \brak{\dfrac{1}{\sqrt{2\pi}}}e^{-\dfrac{1}{2} x^{2}} + \brak{1-\theta}\brak{\dfrac{1}{2}}e^{-\mid x\mid}
\end{align}
Then log of likelihood function is given by

\begin{align}
l(\theta) &= \sum_{i=1}^{i=n} \log f_{\theta}(x_{i})\\
&= \log f_{\theta}(x_{1}) + \log f_{\theta}(x_{2}) \\
&=\log\brak{\theta\brak{\dfrac{1}{\sqrt{2\pi}}}e^{-\dfrac{1}{2}0^{2}}+\brak{1-\theta}\brak{\dfrac{1}{2}}e^{-\mid 0\mid}} \nonumber\\&\hspace{0.75cm}+\log\brak{\theta\brak{\dfrac{1}{\sqrt{2\pi}}} e^{-\dfrac{1}{2}2^{2}}+\brak{1-\theta}\brak{\dfrac{1}{2}}e^{-\mid 2\mid}}
\end{align}
\begin{align}
&=\log\brak{\theta\brak{\dfrac{1}{\sqrt{2\pi}}}+(1-\theta)\brak{\dfrac{1}{2}}}\nonumber\\&\hspace{0.75cm}+\log\brak{\theta\brak{\dfrac{1}{\sqrt{2\pi}}} e^{-2}+\brak{1-\theta}\brak{\dfrac{1}{2}}e^{-2}}\\
&=2\log\brak{\theta\brak{\dfrac{1}{\sqrt{2\pi}}}+\brak{1-\theta}\brak{\dfrac{1}{2}}} -2
\end{align}
Since likelihood $L(\theta) = e^{l(\theta)}$.\\ \\
Likelihood function $L(\theta)$  at $\theta = 0, \frac{1}{2}, 1$ is given by
\begin{enumerate}
    \item At $\theta=0$ \hspace{0.5cm} $L(\theta=0)=\frac{1}{4}e^{-2}=0.0338$\\
    \item At $\theta=1$ \hspace{0.5cm} $L(\theta=1)=\frac{1}{2\pi}e^{-2}=0.0215$\\
    \item At $\theta=\frac{1}{2}$ \hspace{0.2cm}
    $L(\theta=\frac{1}{2})=\brak{\frac{1}{2\sqrt{2\pi}}+\frac{1}{4}}^{2}e^{-2}=0.0273$
\end{enumerate}
Hence the maximum likelihood estimate of $\theta$ is at $\theta=0$

\end{document}
